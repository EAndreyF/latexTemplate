%! Author = eandreyf
%! Date = 2019-09-25

% Preamble
\documentclass[a4paper, 11pt]{article}

% Packages
%%% Работа с русским языком
\usepackage{cmap}					% поиск в PDF
\usepackage{mathtext} 				% русские буквы в формулах
\usepackage[T2A]{fontenc}			% кодировка
\usepackage[utf8]{inputenc}			% кодировка исходного текста
\usepackage[english,russian]{babel}	% локализация и переносы
%%% Дополнительная работа с математикой
\usepackage{amsmath,amsfonts,amssymb,amsthm,mathtools} % AMS
\usepackage{icomma} % "Умная" запятая: $0,2$ --- число, $0, 2$ --- перечисление

%% Номера формул
% \mathtoolsset{showonlyrefs=true} % Показывать номера только у тех формул, на которые есть \eqref{} в тексте.
\newcommand{\myeq}[1]{\stackrel{\mathclap{\mbox{[#1]}}}{=}}

\author{Andrey Evstropov}
\title{
	Разбор семинарских задач \\
	\large ШАД. Теория вероятности
}
\date{2019 г.}

% Document
\begin{document}

\maketitle

\section{Листок 1. Асимптотики}

\subsection{Задача 1}

Найти асимптотику для $\binom{n}{3n}$, при $n \rightarrow \infty$

\textbf{Решение:}
\[
\binom{n}{3n}=\frac{(3n)!}{n!(2n)!}
\]

т.к. $n \to \infty$, то факториалы можем расписать по формуле Стирлинга

\begin{equation*}
\begin{aligned}
\frac{(3n)!}{n!(2n)!}&\sim \frac{
	\sqrt{2\pi3n}{\left(\frac{3n}{e}\right)^{3n}}
}{
	\sqrt{2\pi n}{\left(\frac{n}{e}\right)^{n}}
	\sqrt{2\pi 2n}{\left(\frac{2n}{e}\right)^{2n}}
}=
\\
&= \sqrt{\frac{3}{4\pi n}} \cdot \frac{
	3^{3n}n^{3n}
}{
	n^n 2^{2n}n^{2n}
}  = \frac{
	3^{3n}
}{
	\sqrt{n}2^{2n}
} =
\\
& = \sqrt{\frac{3}{4\pi n}} \cdot \left(\frac{27}{4}\right)^n
\end{aligned}
\end{equation*}

\textbf{Ответ:}
\[
\binom{n}{3n}\sim \sqrt{\frac{3}{4\pi n}} \cdot \left(\frac{27}{4}\right)^n
\]


\section{Листок 2. Асимптотики и подсчет графов}

\subsection{Задача 1}
Подберите функции $f(x)$ и $g(x)$ такие, что $f(x)\sim g(x)$, 
но $e^{f(x)} \neq O(e^{g(x)})$.

\textbf{Ответ:}
\begin{equation*}
\begin{aligned}
f(x)&=x + 1+\frac{1}{x} \\
g(x)&=x
\end{aligned}
\end{equation*}

Тогда:

\begin{equation*}
\lim_{x \to \infty}\frac{f(x)}{g(x)}=\lim_{x \to \infty}\left(1+\frac{1}{x}+\frac{1}{x^2}\right)=1
\end{equation*}

Но:

\begin{equation*}
\lim_{x \to \infty}\frac{e^{f(x)}}{e^{g(x)}}=
	\lim_{x \to \infty}\frac{e^{x +1+\frac{1}{x}}}{e^x}=
	\lim_{x \to \infty}e^{1+\frac{1}{x}}=e\neq 1
\end{equation*}


\subsection{Задача 2}
Найдите асимптотику следующих величин при $n \to \infty$

\indent а) $\frac{(6n)!}{(2n)!(4n-k)!k!}$ при фиксированном $k \in \mathbb{N}$, 
\\

\indent б) $\binom{2n^4}{n^6}$,
\\

\indent в) $\binom{[n/2]}{n}$ (итоговый ответ не должен содержать знака целой части, нужно рассмотреть отдельный случай четного и нечетного n).

\textbf{Решение:}

а) 

т.к. $n \to \infty$, то факториалы можем расписать по формуле Стирлинга

т.к. $k^2 = \bar{o}(n)$, то $\binom{k}{n} \sim \frac{n^k}{k!}$

\begin{equation*}
\begin{aligned}
\frac{(6n)!}{(2n)!(4n-k)!k!}&=
\frac{(6n)!}{(2n)!(4n)!}\binom{k}{4n}&\sim
\\
\sim 
\frac
{
	\sqrt{2\pi 6n}{\left(\frac{6n}{e}\right)^{6n}}
}{
	\sqrt{2\pi 2n}{\left(\frac{2n}{e}\right)^{2n}}
	\sqrt{2\pi 4n}{\left(\frac{4n}{e}\right)^{4n}}
}\cdot \frac{(4n)^k}{k!}
&=
 \sqrt{\frac{3}{8\pi n}} \cdot \frac{
	2^{6n}3^{6n}
}{
2^{10n}
}\cdot \frac{(4n)^k}{k!} 
&=
\\
= 
\sqrt{\frac{3}{8\pi n}} \cdot \frac{
	3^{6n}
}{
	2^{4n}
}\cdot \frac{(4n)^k}{k!} 
&=
 \sqrt{\frac{3}{8\pi n}} \cdot \left(\frac{729}{16}\right)^n\cdot \frac{(4n)^k}{k!}
\end{aligned}
\end{equation*}

б)

Докажем следующее утверждение:
\begin{equation} \label{eq:n4n3}
\begin{aligned}
\binom{k}{n} \sim \frac{n^k}{k!} \cdot e^{-\frac{k(k-1)}{2n} -\frac{(k-1)k(2k-1)}{12n^2}} + O\left(\frac{k^4}{n^3}\right)
\end{aligned}
\end{equation}

Действительно:

\begin{equation*}
\begin{aligned}
\binom{k}{n} &= \frac{n!}{(n-k)!k!} = \frac{n(n-1)\dots (n-k+1)}{k!} 
\myeq{1} 
	 \frac{n^k}{k!} 
	\left(1-\frac{1}{n}\right)
	\left(1-\frac{2}{n}\right) \cdot \dotsc \cdot
	\left(1-\frac{n - k + 1}{n}\right)
\myeq{2}
\\
&\myeq{2}
	\frac{n^k}{k!} 
	e^{
		ln\left(1-\frac{1}{n}\right) +
		ln\left(1-\frac{2}{n}\right) + \dotsc +
		ln\left(1-\frac{n - k + 1}{n}\right)
	}
\myeq{3}
\\
&\myeq{3}
	\frac{n^k}{k!} 
	e^{
		-\frac{1}{n}-\frac{1}{2n^2}+O\left(\frac{1}{n^3}\right)
		-\frac{2}{n}-\frac{4}{2n^2}+O\left(\frac{8}{n^3}\right) + \dotsc
		-\frac{k-1}{n}-\frac{(k-1)^2}{2n^2}+O\left(\frac{(k-1)^3}{n^3}\right)
	}
\myeq{4}
\\
&\myeq{4}
\frac{n^k}{k!} 
e^{
	-\frac{1}{n}\left(1+2+ \dotsc + (k-1)\right)
	-\frac{1}{2n^2}\left(1^2+2^2+ \dotsc + (k-1)^2\right)
	+O\left(\frac{1^3 + 2^3 + \dotsc + (k-1)^3}{n^3}\right)
}
\myeq{5}
\\
&\myeq{5}
\frac{n^k}{k!} 
e^{
	-\frac{k(k-1)}{2n}-\frac{k(k-1)(2k-1)}{12n^2}+O\left(\frac{k^4}{n^3}\right)
}
\end{aligned}
\end{equation*}

\indent [1] Из каждой скобки выносим $n$

\indent [2] Каждую скобку представляем как экспоненту логарифма от этой скобки. Тогда в показателе экспоненты логарифмы суммируются

\indent [3] Расскладываем каждый из логарифмов по формуле Тейлора

\indent [4] Вынесем $\frac{1}{n}$ и $\frac{1}{2n^2}$

\indent [5] Посчитаем сумму прогрессий. Для $O$ нам важен порядок

Таким образом утверждение \eqref{eq:n4n3} доказано.

Рассмотрим наше изначальное сочетание: $\binom{2n^4}{n^6} = \binom{k}{m}$.

Заметим, что для него выполняется условие $k^4 = \bar{o}\left(m^3\right)$

Значит:

\begin{equation*}
\begin{aligned}
\binom{2n^4}{n^6} \sim 
	\frac{\left(n^6\right)^{2n^4}}{(2n^4)!} 
	e^{
		-\frac{2n^4\left(2n^4 -1\right)}{2n^6}
		-\frac{2n^4\left(2n^4 -1\right)\left(4n^4 - 1\right)}{12n^{12}}
	}
\end{aligned}
\end{equation*}

Посчитаем отдельно экспоненту и дробь. Факториал распишем по формуле стирлинга
\begin{equation}
\begin{aligned}
-\frac{2n^4\left(2n^4 -1\right)}{2n^6}
-\frac{2n^4\left(2n^4 -1\right)\left(4n^4 -1\right)}{12n^{12}}
=
\\
=
\frac{-2n^4\left(2n^4 -1\right)6n^6 - 2n^4\left(2n^4 -1\right)\left(4n^4 -1\right)}{12n^{12}}
=
\\
=
\frac{-12n^{10}+6n^6-8n^8+6n^4-1}{6n^8}
\end{aligned}
\end{equation}

\begin{equation}
\begin{aligned}
\frac{\left(n^6\right)^{2n^4}}{(2n^4)!} \sim
\frac{n^{12n^4}}{\sqrt{2\pi 2n^4} \left(\frac{2n^4}{e}\right)^{2n^4}} =
\frac{n^{12n^4}e^{2n^4}}{2n^2\sqrt{\pi} (2n)^{8n^4}} = 
\frac{n^{4n^4-2}e^{2n^4}}{2\sqrt{\pi} 2^{8n^4}}
\end{aligned}
\end{equation}

В результате получим

\begin{equation}
\begin{aligned}
    \frac{n^{4n^4-2}e^{2n^4}}{2\sqrt{\pi} 2^{8n^4}}
    e^{
        \frac{-12n^{10}+6n^6-8n^8+6n^4-1}{6n^8}
    }
\sim
    \frac{n^{4n^4-2}}{2\sqrt{\pi} 2^{8n^4}}
    e^{
        -2n^4 + 2n^2 -\frac{4}{3}
    }
\end{aligned}
\end{equation}

в)

т.к. $n \to \infty$, то факториалы можем расписать по формуле Стирлинга

\begin{equation*}
\begin{aligned}
\binom{[n/2]}{n} = \frac{n!}{[n/2]!(n-[n/2])!} = 
\frac{
	\sqrt{2\pi n}\left(\frac{n}{e}\right)^n
}{
 \sqrt{2\pi [n/2]}
 \left(
 \frac{[n/2]}{e}
 \right)
 ^{[n/2]}
 \sqrt{2\pi (n-[n/2])}
 \left(
 \frac{(n-[n/2])}{e}
 \right)
 ^{(n-[n/2])}
}
=
\\
=
\frac{
	\sqrt{n}n^n
}{
	\sqrt{2\pi [n/2](n-[n/2])} [n/2]^{[n/2]}(n-[n/2])^{n-[n/2])}
}
\end{aligned}
\end{equation*}

Посчитаем данную асимптотику для четных и нечетных $n$

1) n - чётное, тогда можно убрать знак целой части и сокритить дроби 
\[
\frac{\sqrt{n}n^n}{[n/2]\sqrt{2\pi} [n/2]^n} = \frac{2^n\sqrt{2}}{\sqrt{n\pi}}
\]

2) n - нечётное, тогда $[n/2] = \frac{n-1}{2}$
\begin{equation*}
\begin{aligned}
\frac{\sqrt{n}n^n}{
	\sqrt{2\pi\frac{n-1}{2}\frac{n+1}{2}}
	\left(\frac{n-1}{2}\right)^{\frac{n-1}{2}}
	\left(\frac{n+1}{2}\right)^{\frac{n+1}{2}}
}
 = 
\frac{\sqrt{n}(2n)^n}{
 	\sqrt{2\pi(n^2-1)}
 	(n-1)^{\frac{n-1}{2}}
 	(n+1)^{\frac{n+1}{2}}
}
\end{aligned}
\end{equation*}

\textbf{Ответ:}

а) $\frac{\left(\frac{729}{16}\right)^n}{\sqrt{n}}$

б) $\frac{n^{4n^4-2}}{2\sqrt{\pi} 2^{8n^4}}e^{    2n^4 + 2n^2 -\frac{4}{3} }$

в)

\begin{equation*}
\binom{[n/2]}{n} = 
\begin{cases}
	n \text{ - чётно}, & \frac{2^n\sqrt{2}}{\sqrt{n\pi}} \\
	n \text{ - нечётно}, &
	\frac{\sqrt{n}(2n)^n}{
		\sqrt{2\pi(n^2-1)}
		(n-1)^{\frac{n-1}{2}}
		(n+1)^{\frac{n+1}{2}}
	}
\end{cases}
\end{equation*}

\end{document}
