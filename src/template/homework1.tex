\documentclass[12pt,russian,a4paper]{extarticle}

\usepackage[a4paper,left=10mm,right=10mm, top=10mm,bottom=10mm,bindingoffset=0cm]{geometry}
\usepackage{amsfonts,amssymb,amsmath}
\usepackage{nopageno}
\usepackage{cmap}
\usepackage{ifthen}
\usepackage[cp1251]{inputenc}
\usepackage[T2A]{fontenc}
\usepackage[russian]{babel}
\usepackage{tikz}
\usepackage{wrapfig}
\usepackage{float}

\newcommand{\Sum}{\displaystyle\sum\limits}
\newcommand{\Max}{\max\limits}
\newcommand{\Min}{\min\limits}
\newcommand{\fromto}[3]{{#1}=\overline{{#2},\,{#3}}}
\newcommand{\floor}[1]{\left\lfloor{#1}\right\rfloor}
\newcommand{\ceil}[1]{\left\lceil{#1}\right\rceil}
\newcommand{\NN}{\mathbb{N}}
\newcommand{\RR}{\mathbb{R}}
\newcommand{\tild}{\widetilde}
\renewcommand{\le}{\leqslant}
\renewcommand{\ge}{\geqslant}
\renewcommand{\hat}{\widehat}
\renewcommand{\emptyset}{\varnothing}
\renewcommand{\epsilon}{\varepsilon}
\newcommand{\ol}{\overline}

\newcounter{task}

\newcommand{\printscore}[1]{%
\ifthenelse{\equal{#1}{1}}{(1 балл).}{}%
\ifthenelse{\equal{#1}{2}}{(2 балла).}{}%
\ifthenelse{\equal{#1}{3}}{(3 балла).}{}%
\ifthenelse{\equal{#1}{4}}{(4 балла).}{}%
\ifthenelse{\equal{#1}{5}}{(5 баллов).}{}%
\ifthenelse{\equal{#1}{6}}{(6 баллов).}{}%
}

\newcommand{\task}[2]{\par\noindent\stepcounter{task}{\bf Задача~\arabic{task}.~\printscore{#1}} {#2}\vskip 6pt}

\newcommand*{\hm}[1]{#1\nobreak\discretionary{}{\hbox{$#1$}}{}}

%\nofiles
\begin{document}

\centerline{\large \bf Первое домашнее задание по курсу}
\centerline{\large \bf <<Теория вероятностей>>}\bigskip

\task{2}{Дано произвольное натуральное число $k\ge 1$, то есть предполагается, что $k$ --- произвольная заданная константа. Вычислите асимптотику полиномиального коэффициента $P(5n,3n+k,n+k,k)=$ $\frac{(9n+3k)!}{(5n)!(3n+k)!(n+k)!k!}$ при $n\to+\infty$.

\textbf{Внимание!} В задаче требуется найти функцию $f_k(n)$, такую, что $\lim_{n\to+\infty}\frac{f_k(n)}{P(5n,3n+k,n+k,k)}=1$, а не запись биномиального коэффициента в виде $(\alpha+o(1))^n$. Полный балл будет ставиться только в случае, если ответ досчитан до конца, то есть до формулы, которую нельзя упростить. Ответ должен быть записан в виде формулы, не содержащей введённых в процессе решения вспомогательных функций и параметров. За расписывание ассимптотик констант по формуле Стирлинга будет ставиться $0$ баллов - это очень грубая ошибка!}

\task{2}{Найдите асимптотику величины $C_{n^{10}+4n^6}^{n^6}$ при $n\to\infty$.

\textbf{Внимание!} В задаче требуется найти функцию $f(n)$, такую, что $\lim_{n\to+\infty}\frac{f(n)}{C_{n^{10}+4n^6}^{n^6}}=1$, а не запись биномиального коэффициента в виде $(\alpha+o(1))^n$. Полный балл будет ставиться только в случае, если ответ досчитан до конца, то есть до формулы, которую нельзя упростить. Ваш ответ не должен содержать неопределенностей вида $(n^{10}+o(1))^{n^{10}}$ и подобных им - за это Вы получите 0 баллов. Ответ должен быть записан в виде формулы, не содержащей введённых в процессе решения вспомогательных функций и параметров.}

\task{2}{Найдите асимптотику для функции $x(n)=\max\left\{x\in\mathbb{N}:x^{x^2\cdot x!}\leqslant n\right\}.$ Ответ необходимо максимально упростить.}

\task{2}{Посчитайте число графов множество вершин которых совпадает с множеством $\{1,2,\ldots,12\}$ и которые изоморфны графу на рисунке $1$.}

\begin{figure}[H]
\[\begin{tikzpicture}
\node (p1) at (0,0) [scale=0.3,shape=circle,draw=black,fill=white,label=below:$a_{10}$] {};
\node (p2) at (1,0) [scale=0.3,shape=circle,draw=black,fill=white,label=below:$a_{11}$] {};
\node (p3) at (1.5,0.809) [scale=0.3,shape=circle,draw=black,fill=white,label=below:$a_8$] {};
\node (p4) at (1,1.608) [scale=0.3,shape=circle,draw=black,fill=white,label=below:$a_4$] {};
\node (p5) at (0,1.618) [scale=0.3,shape=circle,draw=black,fill=white,label=below:$a_3$] {};
\node (p6) at (-0.5,0.809) [scale=0.3,shape=circle,draw=black,fill=white,label=below:$a_7$] {};
\node (p7) at (2,1.618) [scale=0.3,shape=circle,draw=black,fill=white,label=below:$a_5$] {};
\node (p8) at (3,1.618) [scale=0.3,shape=circle,draw=black,fill=white,label=below:$a_6$] {};
\node (p9) at (-1,1.618) [scale=0.3,shape=circle,draw=black,fill=white,label=below:$a_2$] {};
\node (p10) at (-2,1.618) [scale=0.3,shape=circle,draw=black,fill=white,label=below:$a_1$] {};
\node (p11) at (2,0) [scale=0.3,shape=circle,draw=black,fill=white,label=below:$a_{12}$] {};
\node (p12) at (-1,0) [scale=0.3,shape=circle,draw=black,fill=white,label=below:$a_9$] {};

\draw (p1) -- (p2) -- (p3) -- (p4) -- (p5) -- (p6) -- (p1);
\draw (p4) -- (p7) -- (p8);
\draw (p5) -- (p9) -- (p10);
\draw (p1) -- (p11);
\draw (p2) -- (p12);
\draw (p2) -- (p5);
\draw (0.5,-1.0) node {\tiny{Рис. 1.}};

\end{tikzpicture}\]
\end{figure}

\task{3}{Дано множество $R=\{1,2,\ldots,n\}$. Сколько существует графов, множество вершин которых является подмножеством $R$ и которые являются связными графами на $11$ вершинах, последовательность степеней которых с точностью до перестановки совпадает с последовательностью чисел $3,3,3,3,2,2,2,1,1,1,1$?}

\task{2}{Кидаются четыре симметричные шестигранные пронумерованные кости так, что любые наборы очков на костях равновероятны. Есть четыре события:
\begin{equation*}
\begin{split}
A_1=\{\text{число очков на первой кости} \leqslant \text{числа очков на второй кости}\};\\
A_2=\{\text{число очков на третьей кости} \leqslant \text{числа очков на четвёртой кости}\};\\
A_3=\{\text{число очков на второй кости} \leqslant \text{числа очков на третьей кости}\};\\
A_4=\{\text{число очков на первой кости} \leqslant \text{числа очков на четвёртой кости}\}.
\end{split}
\end{equation*}
Нарисуйте любой орграф зависимостей событий $A_1,$ $A_2,$ $A_3$ и $A_4$ c минимальным числом рёбер.}

\textbf{Внимание!} В следующих трёх задачах, если Вы употребляете слово ``вероятность'', то до этого нужно чётко и однозначно определить вероятностное пространство, так как формулировки самих задач не вероятностны! Помните: слово ``вероятность'' не имеет смысла без задания вероятностного пространства!

\task{2}{Дано семейство различных $k$-элементных подмножеств $\mathcal{S}=\{A_1,\,\ldots,\,A_m\}$ множества $\{v_1,\,\ldots,\,v_n\}$. Назовём элементы $v_i$ и $v_j$ \emph{соседями}, если они вместе входят хотя бы в одно из множеств $A_k$. Пусть у каждого из элементов $v_j$ существует не более чем $2k$ соседей (включая сам $v_j$). Докажите, что элементы $v_1,\,\ldots,\,v_n$ при всех достаточно больших $k$ можно раскрасить пятью красками, так, чтобы никакое подмножество из $\mathcal{S}$ не было одноцветным.}

\task{2}{Пусть $s\ge 100$. Рассмотрим произвольное семейство $\mathcal{S}$ подмножеств множества $\{v_1,\,\ldots,\,v_n\}$, состоящее из множеств мощности $s^2$, такое, что каждый элемент $v_i$  содержится не более, чем в $s$ подмножествах из $\mathcal{S}$. Докажите, что при достаточно большом $s$ элементы $v_1,\,\ldots,\,v_n$ можно раскрасить в $s$ цветов, так, чтобы в каждом множестве из $\mathcal{S}$ присутствовали элементы \emph{всех} этих цветов.}

\task{2}{Матрицу будем называть \emph{занудной}, если все элементы в ней равны одному и тому же числу (неважно, какому). Используя локальную лемму Ловаса, докажите, что если $\sum_{t=0}^a\binom{a}{t}\binom{n-a}{a-t}<\frac{k^{a^2/2}}{2\sqrt{k}}$, то можно расставить числа от $1$ до $k$ в ячейки $n\times n$-матрицы, так, чтобы в ней не оказалось ни одной занудной подматрицы размера $a\times a$.}

\task{2}{Рассмотрим модель случайного графа $G\left(n,\frac{1}{2}\right)$, $n\geqslant 7$. Найдите математическое ожидание (в зависимости от $k\in \mathbb{N}, k\geqslant 6$) числа  неупорядоченных пар \textit{связных компонент} в этом случайном графе, для некоторых $l,m\in\mathbb{N}$ одна из которых является циклом длины $l,$ а вторая --- циклом длины $m$, $l+m = k<n. $ (Ответ вполне может быть записан в виде суммы.)}

\task{2}{Рассмотрим модель случайного графа $G\left(4,\frac{1}{2}\right)$. Найдите вероятность того, что такой случайный граф не содержит треугольников. Треугольником в графе называется клика размера $3$.}


\end{document}
