%! Author = eandreyf
%! Date = 2019-09-25

% Preamble
\documentclass[a4paper, 11pt]{article}

% Packages
%%% Работа с русским языком
\usepackage{cmap}					% поиск в PDF
\usepackage{mathtext} 				% русские буквы в формулах
\usepackage[T2A]{fontenc}			% кодировка
\usepackage[utf8]{inputenc}			% кодировка исходного текста
\usepackage[english,russian]{babel}	% локализация и переносы
%%% Дополнительная работа с математикой
\usepackage{amsmath,amsfonts,amssymb,amsthm,mathtools} % AMS
\usepackage{icomma} % "Умная" запятая: $0,2$ --- число, $0, 2$ --- перечисление

% Рисование графов
\usepackage{tikz}
\usepackage{float}

%% Номера формул
% \mathtoolsset{showonlyrefs=true} % Показывать номера только у тех формул, на которые есть \eqref{} в тексте.
\newcommand{\myeq}[1]{\stackrel{\mathclap{\mbox{[#1]}}}{=}}

\author{Andrey Evstropov}
\title{
	Домашняя работа 1 \\
	\large ШАД. Теория вероятности
}
\date{2019 г.}

\usepackage{hyperref}
\hypersetup{				% Гиперссылки
    unicode=true,           % русские буквы в раздела PDF
    pdftitle={Заголовок},   % Заголовок
    pdfauthor={Автор},      % Автор
    pdfsubject={Тема},      % Тема
    pdfcreator={Создатель}, % Создатель
    pdfproducer={Производитель}, % Производитель
    pdfkeywords={keyword1} {key2} {key3}, % Ключевые слова
    colorlinks=true,       	% false: ссылки в рамках; true: цветные ссылки
    linkcolor=red,          % внутренние ссылки
    citecolor=black,        % на библиографию
    filecolor=magenta,      % на файлы
    urlcolor=cyan           % на URL
}

% Собирать только указанный файл. Чтобы собрать всё, нужно закомментировать следующую строку
\includeonly{tasks/task1}

% Document
\begin{document}

\maketitle

\tableofcontents

% Подключаем каждое решение в отдельном файле
% !TEX root= ../hw1.tex

\section{Задача 1}

Найти асимптотику для $\frac{(9n+3k)!}{(5n)!(3n + k)!(n+k)!k!}$, при $n \rightarrow +\infty$

\textbf{Решение:}

\begin{equation}
\begin{aligned}
    \frac{
        (9n+3k)!
    }{
        (5n)!(3n + k)!(n+k)!k!
    }
&\myeq{1}
\\
\myeq{1}
\end{aligned}
\end{equation}


\begin{equation}
\begin{aligned}
    \left(
        \frac{
            n+1
        }{
            k
        }
    \right)
    ^{5m}
=
    \left(
        \frac{2}{4}
    \right)^{6}
    \left(
        1+
        \frac{
            \frac{2}{3}r
        }{2q}
    \right)^{7n}
\sim
    \left(
        \frac{2}{4}
    \right)^{1}
    e^{
        \frac{3}{5}8
    }
\end{aligned}
\end{equation}

\textbf{Ответ:}


\end{document}
